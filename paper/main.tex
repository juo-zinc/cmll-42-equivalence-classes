\documentclass[11pt]{article}

\usepackage[margin=1in]{geometry}
\usepackage{amsmath,amssymb,amsthm}
\usepackage{mathtools}
\usepackage[hidelinks]{hyperref}

% --- Theorem environments ---
\theoremstyle{plain}
\newtheorem{theorem}{Theorem}
\newtheorem{lemma}{Lemma}
\newtheorem{proposition}{Proposition}

\theoremstyle{definition}
\newtheorem{definition}{Definition}
\newtheorem{remark}{Remark}

% --- Commands ---
\newcommand{\Z}{\mathbb{Z}}
\newcommand{\Zthree}{\Z/3\Z}
\newcommand{\sgn}{\operatorname{sgn}}
\newcommand{\id}{\mathrm{id}}

\title{Why CMLL Has 42 Cases: Reachability, Parity, and Burnside's Lemma}
\author{Jamie Luo}
\date{December 14, 2025}

\begin{document}
\maketitle

\begin{abstract}
\noindent
We give a group-theoretic derivation of the number of CMLL cases arising in
the Roux method for the Rubik’s Cube. Under the standard CMLL constraints—fixing the bottom-layer corners and a
specified set of six edges—we show that the space of top-layer corner states
compatible with the cube-group reachability invariants has cardinality
$4!\cdot 3^3 = 648$.
In particular, no even-permutation restriction is imposed on the corner
permutation, as parity may be absorbed by the remaining unsolved edges.

\noindent
We then quotient this corner state space by a natural symmetry
action and apply Burnside's lemma to compute the resulting orbit count.
The calculation yields exactly $42$ equivalence classes.
Here ``admissible'' refers to global reachability (existence of a globally reachable
cube state satisfying the fixed-piece constraints), not reachability via move sequences
that preserve the Roux blocks throughout.
\end{abstract}

\section{Introduction}

In the Roux method for solving the Rubik's Cube, the \emph{CMLL} step completes the
four top-layer corners while preserving two already-built blocks.
A recurring point of confusion is whether the corner-permutation parity forces one
to ``divide by $2$'' (i.e.\ restrict to even corner permutations) when counting CMLL cases.

\noindent The resolution is that \emph{CMLL is typically not the final step of the Roux solve}:
several edges (notably in the M-slice) remain unsolved and can absorb permutation parity.
Consequently, the set of attainable top-corner states under the CMLL constraints
has size $4!\cdot 3^3 = 648$ rather than $12\cdot 3^3 = 324$.
After establishing this state space, we then classify these $648$ states up to a standard
symmetry relation and obtain exactly $42$ equivalence classes via Burnside's lemma.

\noindent We proceed in two steps: first we justify the $648$ admissible top-corner states under the usual CMLL constraints; then we quotient by the standard symmetries and apply Burnside's lemma to obtain $42$.

\section{Preliminaries and State Model}

We model the Rubik's Cube using the standard \emph{cubie model}. Let
\[
C=\{1,\dots,8\} \quad\text{(corners)}, \qquad
E=\{1,\dots,12\} \quad\text{(edges)}.
\]
A cube state is represented by a quadruple
\[
(\sigma_c, t_c, \sigma_e, t_e),
\]
where:
\begin{itemize}
  \item $\sigma_c \in S_8$ is the corner permutation;
  \item $t_c=(t_1,\dots,t_8)\in (\Z/3\Z)^8$ is the corner orientation vector;
  \item $\sigma_e \in S_{12}$ is the edge permutation;
  \item $t_e=(f_1,\dots,f_{12})\in (\Z/2\Z)^{12}$ is the edge orientation vector.
\end{itemize}
We take the solved cube as the identity state.

\begin{remark}
The particular conventions used to assign corner/edge orientations do not
affect the statements below, provided they satisfy the standard global
constraints described in Theorem~\ref{thm:reachability}.
\end{remark}

\section{Classical Reachability Conditions}

The following theorem is classical in the mathematical analysis of the Rubik's
Cube.

\begin{theorem}[Reachability Invariants]\label{thm:reachability}
A state $(\sigma_c,t_c,\sigma_e,t_e)$ is reachable from the solved cube by legal
face turns if and only if the following three conditions hold:
\begin{enumerate}
  \item \textbf{Parity:} $\sgn(\sigma_c) = \sgn(\sigma_e)$;
  \item \textbf{Corner orientation:} $\sum_{i=1}^8 t_i \equiv 0 \pmod{3}$;
  \item \textbf{Edge orientation:} $\sum_{j=1}^{12} f_j \equiv 0 \pmod{2}$.
\end{enumerate}
\end{theorem}

\begin{remark}
The necessity direction can be verified by checking that each generator
$U,D,L,R,F,B$ preserves these three quantities. The sufficiency direction is
nontrivial in general. In this note we only require Theorem~\ref{thm:reachability}
as a black box and give a fully constructive argument in the constrained setting
below.
\end{remark}

\section{Constrained Setting and Statement}

Fix the following constraints, motivated by partially built cube states:

\begin{itemize}
  \item The \emph{bottom four corner positions} (say positions $5,6,7,8$) are solved:
  \[
  \sigma_c(i)=i,\quad t_i=0 \qquad (i=5,6,7,8).
  \]
  \item A prescribed subset $B\subset E$ of \emph{six edges} is solved in both
  position and orientation:
  \[
  \sigma_e(k)=k,\quad f_k=0 \qquad (k\in B).
  \]
  Let $R:=E\setminus B$ be the remaining set of \emph{unconstrained} edges, so
  $|R|=6$.
\end{itemize}

\noindent Let $\tau := \sigma_c|_{\{1,2,3,4\}} \in S_4$ denote the induced permutation of
the top four corner positions, and let $(t_1,t_2,t_3,t_4)$ denote their
orientations.

\begin{definition}[Admissible top-corner state]
A pair $(\tau,(t_1,t_2,t_3,t_4))$ with $\tau\in S_4$ and
$(t_1,t_2,t_3,t_4)\in(\Z/3\Z)^4$ satisfying $t_1+t_2+t_3+t_4\equiv 0\pmod 3$
is called \emph{admissible} if it occurs as the top-corner component of some cube state
that satisfies the fixed bottom corners and fixed edge set $B$ and is globally reachable
from the solved cube.
\end{definition}

\begin{theorem}[Corner state space under six fixed edges]\label{thm:main}
Under the constraints above, every top-corner configuration consisting of an
arbitrary permutation $\tau\in S_4$ and an orientation vector
$(t_1,t_2,t_3,t_4)\in (\Z/3\Z)^4$ satisfying
\[
t_1+t_2+t_3+t_4 \equiv 0 \pmod{3}
\]
is admissible. Consequently, the number of admissible top-corner states is
$4!\cdot 3^3 = 648$.
\end{theorem}



\begin{remark}
Theorem~\ref{thm:main} is a global consistency statement: it asserts existence of a globally
reachable cube state realizing the specified top-corner data under the fixed-piece constraints.
It does not address reachability via move sequences that preserve the Roux blocks throughout.
\end{remark}


\section{Proof of Theorem~\ref{thm:main}}

We prove realizability by constructing compatible edge data that satisfy the
reachability invariants of Theorem~\ref{thm:reachability}.

\subsection{Corner orientation constraint}

\begin{lemma}\label{lem:corner-orient}
If $(t_1,t_2,t_3,t_4)$ satisfies $\sum_{i=1}^4 t_i \equiv 0 \pmod{3}$ and
$t_5=\cdots=t_8=0$, then $\sum_{i=1}^8 t_i \equiv 0 \pmod{3}$.
\end{lemma}

\begin{proof}
Immediate:
\[
\sum_{i=1}^8 t_i = \sum_{i=1}^4 t_i + \sum_{i=5}^8 t_i \equiv 0 + 0 \equiv 0
\pmod{3}.
\]
\end{proof}

\subsection{Edge orientation constraint}

\begin{lemma}\label{lem:edge-orient}
There exists an edge orientation assignment $(f_1,\dots,f_{12})$ such that
$f_k=0$ for all $k\in B$ and $\sum_{j=1}^{12} f_j \equiv 0 \pmod{2}$.
\end{lemma}

\begin{proof}
Set $f_k=0$ for $k\in B$. On the remaining set $R$ (which has size $6$), freely
choose the values of $f$ on any five elements of $R$. Define the sixth value so
that the total sum over $R$ is $0$ modulo $2$. Then
$\sum_{j=1}^{12} f_j = \sum_{k\in B} f_k + \sum_{r\in R} f_r \equiv 0+0 \equiv 0
\pmod{2}$.
\end{proof}

\subsection{Parity matching via the remaining edges}

\begin{lemma}\label{lem:parity}
For any $\tau\in S_4$ there exists a permutation $\pi\in S_6$ such that
$\sgn(\pi)=\sgn(\tau)$.
\end{lemma}

\begin{proof}
If $\sgn(\tau)=+1$, take $\pi=\id$. If $\sgn(\tau)=-1$, take $\pi$ to be any
transposition in $S_6$, e.g.\ $(1\ 2)$ in a fixed labeling of the $6$ elements.
A transposition is an odd permutation, hence has sign $-1$. Therefore
$\sgn(\pi)=\sgn(\tau)$ in all cases.
\end{proof}

\subsection{Assembling a reachable full state}

\begin{proof}[Proof of Theorem~\ref{thm:main}]
Fix an arbitrary $\tau\in S_4$ and an orientation vector $(t_1,t_2,t_3,t_4)$
satisfying $\sum_{i=1}^4 t_i \equiv 0 \pmod{3}$. Define:
\begin{itemize}
  \item $\sigma_c\in S_8$ by setting $\sigma_c|_{\{1,2,3,4\}}=\tau$ and
  $\sigma_c(i)=i$ for $i=5,6,7,8$;
  \item $t_c=(t_1,\dots,t_8)$ by setting $t_5=\cdots=t_8=0$;
  \item choose $\pi\in S_6$ as in Lemma~\ref{lem:parity};
  \item define $\sigma_e\in S_{12}$ by $\sigma_e(k)=k$ for $k\in B$ and
  $\sigma_e|_{R}=\pi$ (after identifying $R$ with a $6$-element set);
  \item choose $t_e=(f_1,\dots,f_{12})$ as in Lemma~\ref{lem:edge-orient}.
\end{itemize}

Then:
\begin{enumerate}
  \item By Lemma~\ref{lem:corner-orient}, $\sum_{i=1}^8 t_i \equiv 0 \pmod{3}$.
  \item By Lemma~\ref{lem:edge-orient}, $\sum_{j=1}^{12} f_j \equiv 0 \pmod{2}$.
  \item For parity, since $\sigma_c$ fixes corners $5$--$8$, we have
  $\sgn(\sigma_c)=\sgn(\tau)$. Since $\sigma_e$ fixes all edges in $B$, we have
  $\sgn(\sigma_e)=\sgn(\pi)$. By Lemma~\ref{lem:parity},
  \[
  \sgn(\sigma_c)=\sgn(\tau)=\sgn(\pi)=\sgn(\sigma_e).
  \]
\end{enumerate}
Thus all conditions of Theorem~\ref{thm:reachability} hold, so the constructed state
$(\sigma_c,t_c,\sigma_e,t_e)$ is globally reachable from the solved cube.
By construction it satisfies the fixed bottom corners and the fixed edge set $B$,
and its top-corner component is $(\tau,(t_1,t_2,t_3,t_4))$.
This proves admissibility of every such top-corner configuration.

\noindent
For the count: $\tau$ can be chosen arbitrarily in $S_4$, giving $4!=24$ choices.
For orientations, the constraint $\sum_{i=1}^4 t_i\equiv 0\pmod 3$ leaves $3^3$
choices (choose three entries freely and determine the fourth).
Hence the total is $4!\cdot 3^3=648$.
\end{proof}

\section{Remarks and Variants}

\begin{remark}
If instead \emph{all} $12$ edges are required solved, then $\sigma_e=\id$ forces
$\sgn(\sigma_c)=+1$. With bottom corners fixed, the top-corner permutation must
lie in $A_4$ (size $12$), yielding $12\cdot 3^3=324$ reachable top-corner states.
\end{remark}

\section{Orbit-counting setup: the state space and symmetries}

We consider four \emph{positions} arranged cyclically, corresponding to the four
corner positions in the top layer. For definiteness, label the positions by
\[
P := \{0,1,2,3\},
\]
with the cyclic order $0\to 1\to 2\to 3\to 0$.

\begin{definition}[Permutation data]\label{def:tau}
Let $\mathfrak{S}(P)\cong S_4$ be the symmetric group on $P$. We represent a
corner permutation state by a bijection
\[
\tau: P \to P,
\]
interpreted as: \emph{the corner cubie labeled $x\in P$ occupies position
$\tau^{-1}(x)$}, equivalently \emph{the cubie occupying position $p$ is
$\tau(p)$}. Concretely, $\tau$ is a $4$-tuple $(\tau(0),\tau(1),\tau(2),\tau(3))$
with distinct entries in $\{0,1,2,3\}$.
\end{definition}

\begin{definition}[Orientation data]\label{def:t}
Let $\mathbb{Z}_3 := \mathbb{Z}/3\mathbb{Z}$. An orientation state is a $4$-tuple
\[
t=(t_0,t_1,t_2,t_3)\in \Z_3^{\,4}
\]
subject to the corner twist constraint
\[
t_0+t_1+t_2+t_3 \equiv 0 \pmod 3.
\]
\end{definition}

\begin{definition}[Corner state space]
Define
\[
X := \{\,(\tau,t)\mid \tau\in S(P),\ t \in \Z_3^{\,4},\ t_0+t_1+t_2+t_3\equiv 0 \pmod 3\,\}.
\]
\end{definition}

\begin{proposition}[Cardinality of $X$]\label{prop:|X|}
We have $|X|=4!\cdot 3^3=648$.
\end{proposition}

\begin{proof}
There are $4!=24$ choices for $\tau$. For $t$, choose $(t_0,t_1,t_2)\in Z_3^{\,3}$ arbitrarily (giving $3^3$ choices),
and then $t_3$ is uniquely determined by the constraint:
\[
t_3 \equiv -(t_0+t_1+t_2)\pmod 3.
\]
Hence $|X|=24\cdot 27=648$.
\end{proof}

\section{The symmetry group acting on $X$}

We define two types of transformations of $X$:
\begin{enumerate}
  \item position symmetries of the $4$-cycle (a dihedral action on positions),
  \item an involution that inverts all corner twists.
\end{enumerate}
Together these generate a group of order $16$.

\subsection{Dihedral symmetries on positions}

Let $D_4$ be the dihedral group of the square, realized here as the group of
symmetries of the cyclically ordered set $P$.

\begin{definition}[The group $D_4$]\label{def:D4}
Let $r$ be the rotation $r(p)=p+1\ (\mathrm{mod}\ 4)$, and let $s$ be the
reflection $s(0)=0$, $s(1)=3$, $s(2)=2$, $s(3)=1$. Let
\[
D_4 := \langle r,s \mid r^4=e,\ s^2=e,\ srs=r^{-1}\rangle.
\]
We view each $g\in D_4$ as a permutation of the position set $P$.
\end{definition}

\begin{definition}[Action of $D_4$ on $X$]\label{def:D4action}
For $g\in D_4$ and $(\tau,t)\in X$, define
\[
g\cdot(\tau,t) := (\tau\circ g^{-1},\ t\circ g^{-1}),
\]
where $(\tau\circ g^{-1})(p)=\tau(g^{-1}(p))$ and similarly
$(t\circ g^{-1})(p)=t_{g^{-1}(p)}$.
\end{definition}

\begin{remark}
This is the standard \emph{relabeling of positions} action: applying $g$ rotates
or reflects the top layer positions, so the cubie and twist values attached to a
position $p$ are transported from the previous position $g^{-1}(p)$.
\end{remark}

\subsection{Twist inversion}

\begin{definition}[Twist inversion]\label{def:iota}
Define $\iota:X\to X$ by
\[
\iota(\tau,t) := (\tau, -t),
\qquad\text{where } (-t)_p := -t_p\in \mathbb Z_3.
\]
\end{definition}

\begin{lemma}\label{lem:iota-well-defined}
$\iota$ is well-defined on $X$ and satisfies $\iota^2=\mathrm{id}_X$.
\end{lemma}

\begin{proof}
If $\sum_{p\in P} t_p \equiv 0\pmod 3$, then
\[
\sum_{p\in P} (-t_p) \equiv -\sum_{p\in P} t_p \equiv 0 \pmod 3,
\]
so $-t$ also satisfies the constraint and $\iota(\tau,t)\in X$. Also,
$-(-t)=t$, hence $\iota^2=\mathrm{id}_X$.
\end{proof}

\begin{lemma}\label{lem:commute}
For every $g\in D_4$ we have $\iota(g\cdot x)=g\cdot\iota(x)$ for all $x\in X$.
\end{lemma}

\begin{proof}
Let $x=(\tau,t)$. Then
\[
\iota(g\cdot(\tau,t))
=\iota(\tau\circ g^{-1},\, t\circ g^{-1})
=(\tau\circ g^{-1},\, -(t\circ g^{-1})).
\]
On the other hand
\[
g\cdot \iota(\tau,t)
=g\cdot(\tau,-t)
=(\tau\circ g^{-1},\, (-t)\circ g^{-1})
=(\tau\circ g^{-1},\, -(t\circ g^{-1})).
\]
Thus they agree.
\end{proof}

\subsection{The full action group}

\begin{definition}[The action group $K$]\label{def:K}
Let $C_2=\langle \iota\rangle$ be the order-$2$ group generated by twist inversion.
By Lemma~\ref{lem:commute}, the actions of $D_4$ and $C_2$ commute, so we obtain a
well-defined action of the direct product
\[
K := D_4\times C_2
\]
on $X$, given by
\[
(g,\epsilon)\cdot(\tau,t) := g\cdot(\tau,\epsilon t),
\quad\text{where }\epsilon\in\{+1,-1\}\text{ and }(-1)t:=-t.
\]
\end{definition}

\begin{proposition}\label{prop:Korder}
$|K|=16$.
\end{proposition}

\begin{proof}
$|D_4|=8$ and $|C_2|=2$, so $|K|=|D_4||C_2|=16$.
\end{proof}

\section{Orbit count via Burnside's lemma}

\begin{definition}[Equivalence classes]
Define an equivalence relation on $X$ by
\[
x\sim y \iff \exists k\in K \text{ such that } y=k\cdot x.
\]
The equivalence classes are precisely the $K$-orbits in $X$.
\end{definition}

\begin{theorem}[Orbit count]\label{thm:42}
The action of $K$ on $X$ has exactly $42$ orbits.
\end{theorem}

The proof uses the classical orbit-counting tool.

\begin{theorem}[Burnside's lemma]\label{thm:burnside}
Let a finite group $K$ act on a finite set $X$. Then the number of orbits is
\[
|X/K| = \frac{1}{|K|}\sum_{k\in K} |\mathrm{Fix}(k)|,
\]
where $\mathrm{Fix}(k)=\{x\in X: k\cdot x=x\}$.
\end{theorem}

\subsection{Fixed-point counts}

We now compute $|\mathrm{Fix}(k)|$ for each $k\in K$.

\begin{lemma}\label{lem:fix-identity}
For the identity element $e\in K$, we have $|\mathrm{Fix}(e)|=|X|=648$.
\end{lemma}

\begin{proof}
Trivial: $e\cdot x=x$ for all $x\in X$.
\end{proof}

\begin{lemma}\label{lem:fix-iota}
For the element $\iota\in K$ (i.e.\ $(e,-1)$ in $D_4\times C_2$), we have
$|\mathrm{Fix}(\iota)|=24$.
\end{lemma}

\begin{proof}
A state $(\tau,t)\in X$ is fixed by $\iota$ iff
\[
(\tau,t)=\iota(\tau,t)=(\tau,-t),
\]
which holds iff $t=-t$ in $(\Zthree)^4$. In $\Zthree$, the equation $u=-u$ implies
$2u\equiv 0\pmod 3$, hence $u\equiv 0\pmod 3$. Therefore $t=(0,0,0,0)$ is the only
orientation vector fixed by $\iota$.

\noindent With $t$ forced to be the zero vector, $\tau$ may be chosen arbitrarily in $S_4$,
giving $4!=24$ fixed states.
\end{proof}

\begin{lemma}\label{lem:fix-nontrivial-D4}
Let $g\in D_4$ be nontrivial ($g\neq e$). Then
\[
|\mathrm{Fix}((g,+1))|=0.
\]
\end{lemma}

\begin{proof}
Suppose $(\tau,t)\in X$ is fixed by $(g,+1)$, i.e.\ by the position relabeling action
of $g$. By Definition~\ref{def:D4action}, the condition $(g,+1)\cdot(\tau,t)=(\tau,t)$
implies in particular
\[
\tau\circ g^{-1}=\tau.
\]
Equivalently, $\tau(g^{-1}(p))=\tau(p)$ for all $p\in P$.

\noindent Since $g\neq e$ as a permutation of $P$, there exists some $p\in P$ with
$g^{-1}(p)\neq p$. Applying the equality above at this $p$ yields
\[
\tau(g^{-1}(p))=\tau(p)
\quad\text{with}\quad g^{-1}(p)\neq p.
\]
But $\tau$ is a bijection $P\to P$ (Definition~\ref{def:tau}), so it cannot take
the same value at two distinct inputs. This is a contradiction. Hence no such
$(\tau,t)$ exists, and the fixed-point set is empty.
\end{proof}

\begin{lemma}\label{lem:fix-nontrivial-D4-iota}
Let $g\in D_4$ be nontrivial ($g\neq e$). Then
\[
|\mathrm{Fix}((g,-1))|=0.
\]
\end{lemma}

\begin{proof}
If $(\tau,t)$ is fixed by $(g,-1)$, then by Definition~\ref{def:K} and
Definition~\ref{def:D4action} we have
\[
(\tau,t)=(g,-1)\cdot(\tau,t)=g\cdot(\tau,-t)=(\tau\circ g^{-1},\, (-t)\circ g^{-1}).
\]
In particular, $\tau=\tau\circ g^{-1}$. The same bijectivity argument as in
Lemma~\ref{lem:fix-nontrivial-D4} yields a contradiction whenever $g\neq e$.
Thus the fixed-point set is empty.
\end{proof}

\subsection{Conclusion of the orbit count}

\begin{proof}[Proof of Theorem~\ref{thm:42}]
By Proposition~\ref{prop:Korder}, $|K|=16$. By Lemmas~\ref{lem:fix-identity},
\ref{lem:fix-iota}, \ref{lem:fix-nontrivial-D4}, and \ref{lem:fix-nontrivial-D4-iota},
the fixed-point counts satisfy:
\[
|\mathrm{Fix}(k)|=
\begin{cases}
648, & k=e,\\
24,  & k=\iota,\\
0,   & \text{for all other }k\in K.
\end{cases}
\]
Therefore Burnside's lemma (Theorem~\ref{thm:burnside}) gives
\[
|X/K|
=\frac{1}{16}\Bigl(648+24\Bigr)
=\frac{672}{16}
=42.
\]
Hence the action has exactly $42$ orbits.
\end{proof}

\section{Interpretation (optional)}

\begin{remark}
The equivalence relation generated by the $D_4$ action identifies states that
differ only by relabeling of the four top-layer corner \emph{positions} via
rotations and reflections of the layer. The twist-inversion involution $\iota$
identifies states differing by reversing the direction of each corner twist
simultaneously. Theorem~\ref{thm:42} states that, under these identifications,
the $648$ admissible states fall into precisely $42$ equivalence classes.
\end{remark}

\end{document}
